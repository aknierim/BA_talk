% package settings
\unimathsetup{
  math-style=ISO,
  bold-style=ISO,
  sans-style=italic,
  nabla=upright,
  partial=upright,
  mathrm=sym,
}

\sisetup{
  separate-uncertainty=true,
  per-mode=reciprocal,
  output-decimal-marker={.},
  range-phrase = \text{--},
}
\DeclareSIUnit\crab{Crab}

% tikz settings
\usetikzlibrary{overlay-beamer-styles,calc,tikzmark,decorations.pathreplacing}
\tikzset{fontscale/.style = {font=\relsize{#1}}}

\setmathfont{XITS Math}[range={scr, bfscr}]
\setmathfont{XITS Math}[range={cal, bfcal}, StylisticSet=1]

% checks if argument is empty
\def \ifempty#1{\def\temp{#1} \ifx\temp\empty }

% This adds a circle with a picture of your choice in it.
% Usage:
% \roundpic[<optional arguments>, eg. xshift or yshift]\
% {<radius of the cirlce [cm]>}{<picture width [cm]>}{<path_to_picture>}{x pos}{y pos}{label}
\newcommand{\roundpic}[7][]{%
  \ifempty{#7}
    \node [circle, draw, color=tugreen, minimum width = #2,
      path picture = {
        \node [#1] at (path picture bounding box.center) {
          \includegraphics[width=#3]{#4}};
      }] at (#5,#6) {};
  \else
    \node [circle, draw, color=tugreen, minimum width = #2,
      path picture = {
        \node [#1] at (path picture bounding box.center) {
          \hyperlink{#7}{\includegraphics[width=#3]{#4}}};
      }] at (#5,#6) {};
  \fi
}%

% Comment this out to represent vectors with an arrow on top.
% Uncomment this to represent vectors as bold symbols.
\renewcommand{\vec}[1]{\mathbf{#1}}


\DeclareMathOperator{\tp}{tp}
\DeclareMathOperator{\fp}{fp}
\DeclareMathOperator{\fn}{fn}
\DeclareMathOperator{\tn}{tn}
\DeclareMathOperator{\recall}{recall}
\DeclareMathOperator{\precision}{precision}
\DeclareMathOperator{\tpr}{TPR}
\DeclareMathOperator{\fpr}{FPR}
\DeclareMathOperator{\tnr}{TNR}
\DeclareMathOperator{\fnr}{FNR}
\DeclareMathOperator{\acc}{ACC}
\DeclareMathOperator{\ba}{BA}
\DeclareMathOperator{\eff}{Eff}
