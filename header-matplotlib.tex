\usepackage{amsmath}
\usepackage{amssymb}
\usepackage{mathtools}
\usepackage{fontspec}
\usepackage{nicefrac}
\usepackage{xfrac}
\usepackage{xspace}

\usepackage[
  math-style=ISO,    % ┐
  bold-style=ISO,    % │
  sans-style=italic, % │ follow ISO standard
  nabla=upright,     % │
  partial=upright,   % ┘
  warnings-off={           % ┐
    mathtools-colon,       % │ disable some warnings
    mathtools-overbracket, % │
  },                       % ┘
]{unicode-math}

% font settings
\setmainfont{Fira Sans}
\setmathfont{Fira Math}
\setmathfont{XITS Math}[range={scr, bfscr}]
\setmathfont{XITS Math}[range={cal, bfcal}, StylisticSet=1]

% settings for decimals and units
\usepackage[
  separate-uncertainty=true,
  per-mode=fraction,
]{siunitx}
\DeclareSIUnit\year{yr}
\DeclareSIUnit\deg{deg}
\DeclareSIUnit\pe{p.e.}

\usepackage[
  version=4,
  math-greek=default,
  text-greek=default,
]{mhchem}


% \usepackage[xindy,toc]{glossaries}
% \newcommand{\magic}{MAGIC\xspace}
\newcommand{\cta}{CTA\xspace}
\newcommand{\ctapipe}{\texttt{ctapipe}\xspace}
